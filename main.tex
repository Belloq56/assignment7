\documentclass{article}
\usepackage{graphicx} % Required for inserting images
\usepackage{amsmath}

\title{Assignment 7}
\author{Stefano Migneco}
\date{}

\begin{document}

\maketitle

\section{Numerical Differentiation}
There are several methods to calculating numerical derivatives. The first of these is forward differencing, where you start with the first-order Taylor series for (x+h), a step forward:
\begin{align}
    f(x+h)=f(x)+ \frac {f'(x)h} {1!}\\
    f'(x)=\frac {f(x+h)-f(x)} {h}
\end{align}
This same process can be repeated for backwards differencing, giving the result of $f'(x)=\frac {f(x)-f(x-h)} {h}$\\
By taking the difference of both a step forward and backward, and including the second order term, you can arrive at the central differencing scheme:
\begin{align}
    f(x+h)=f(x)+ \frac {f'(x)h} {1!} +\frac{f''(x)h^2}{2!}\\
    f(x-h)=f(x)- \frac {f'(x)h} {1!} +\frac{f''(x)h^2}{2!}\\
    f(x+h)-f(x-h) = f'x(h)+f'(x)h\\
    f'(x) =\frac{f(x+h)-f(x-h)}{2h}
\end{align}
This can be further expanded to a higher-order scheme, such as the five-point stencil approximation, where you take two points on either side of the point you are evaluating:
\begin{align}
    f(x+h)=f(x)+ \frac {f'(x)h} {1!} +\frac{f''(x)h^2}{2!}+\frac{f'''(x)h^3}{3!}\\
    f(x-h)=f(x)- \frac {f'(x)h} {1!} +\frac{f''(x)h^2}{2!}-\frac{f'''(x)h^3}{3!}\\
    f(x+2h)=f(x)+ \frac {2f'(x)h} {1!} +\frac{4f''(x)h^2}{2!}+\frac{8f'''(x)h^3}{3!}\\
    f(x-2h)=f(x)- \frac {2f'(x)h} {1!} +\frac{4f''(x)h^2}{2!}-\frac{8f'''(x)h^3}{3!}\\
    E1=f(x+h)-f(x-h)=\frac {2f'(x)h} {1!} + \frac {f'''(x)h} {3}\\
    E2=f(x+2h)-f(x-2h)=\frac {4f'(x)h} {1!}+\frac{8f'''(x)h^3}{3}\\
    8E1-E2=\frac {12f'(x)h} {1!}\\
    f'(x)=\frac{8f(x+h)-8f(x-h)-f(x+2h)+f(x-2h)}{12h}
\end{align}


\section{Electric Potential}
This was the most difficult part of the assignment. In this, a dict of 50 random strength point charges at 50 random locations was created. I extracted the dict into two separate arrays, and then summed each 2D array of voltages created by looping through the dict to create a map. Then, I used a quiver plot to create a display of the electric field.
\begin{figure}[h]
    \centering
    \includegraphics[width=0.5\linewidth]{contour.png}
    \caption{A sample plot of the charges and electric field}
    \label{fig:potential}
\end{figure}


\section{Harmonic Oscillator}
In this part of the assignment, I programmed two ODEs to represent damped oscillation. There were 4 cases: 0 damping, critical damping at exactly $\frac{4k}{m}$, under-damping at less than $\frac{4k}{m}$, and over-damping at greater than $\frac{4k}{m}$.\\
\begin{figure}[htp!]
    \centering
    \includegraphics[width=0.5\linewidth]{C=0.png}
    \caption{Oscillation with 0 damping}
    \label{fig:C=0}
\end{figure}

\begin{figure}[htp!]
    \centering
    \includegraphics[width=0.5\linewidth]{C=4km.png}
    \caption{No oscillation due to critical damping}
    \label{fig:C=critical}
\end{figure}
\begin{figure}[htp!]
    \centering
    \includegraphics[width=0.5\linewidth]{C less than 4km.png}
    \caption{Oscillation with under-damping}
    \label{fig:C<critical}
\end{figure}
\begin{figure}[htp!]
    \centering
    \includegraphics[width=0.5\linewidth]{C greater than 4km.png}
    \caption{No oscillation due to over-damping}
    \label{fig:C>critical}
\end{figure}

\section{Cycling}
In this question, we were given an ODE representing acceleration as a function of power, mass, and velocity. We were then tasked with solving the ODE over a span of 200 seconds, given initial velocity of 4 m/s, mass of 70kg, and power of 400W.
\begin{figure}[h!]
    \centering
    \includegraphics[width=0.5\linewidth]{bicycle.png}
    \label{fig:bicycle}
\end{figure}
\end{document}
